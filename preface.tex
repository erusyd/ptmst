
\chapter*{序\quad 言}
\addcontentsline{toc}{chapter}{序言}
概率论与数理统计是全国高等院校数学系与统计系的基础课程。这门课的任务是以丰富的背景、巧妙的思维和有趣的结论吸引读者,使学生在浓厚的兴趣中学习和掌握概率论与数理统计的基本概念、基本方法和基本理论。我们正是抱着这样的心愿编写这本教科书,并努力去实现它。很幸运,2003年该书先后被列入“国家级十五规划教材”和“高等教育百门精品课程教材建设项目”。这使我们信心倍增,同时也深感责任重大,定要同心协力编写好此书,以适应祖国日益发展的经济形势的需要。

本书内容为八章,前四章为概率论,后四章为数理统计。在编写上作了一些尝试,我们把随机变量的定义分两步完成,其直观定义在第一章就出现了,用来表示事件,较为严格的定义在第二章中完成。这样可使学生对随机变量有了校具体又完整的概念。在随机变量层次上,我们更强调分布的概念。另外在概率定义上,我们采用了公理化系统,而把频率、古典概率、几何概率和主观概率作为确定概率的四种方法。在统计部分,我们尽量从数据出发提出问题和研究问题,对总体、抽样分布、检验的拒绝域等概念的叙述都作了一些改进,增加了描述性统计的基本内容和贝叶斯统计初步,让学生能较为全面地认识统计。另外对分位数、检验的$p$值、零概率事件(几乎处处)和渐近分布等都作了较为详尽和具体的叙述。在叙述中我们尽力做到图文并茂,全书共有图100多幅,相信这对内容的理解会有帮助。

作为概率论与数理统计的入门书,我们不想一进门就把学生引入数学天堂,而是在“野外”先浏览概率统计的各种风景之后,再进入数学天堂,使各种概念和定理成为有源之水、有本之木。可使学生愁到读此书的趣味,感到与读数学教科书有不同的味道。当然我们也十分注意从偶然性中提炼出来的一些规律性的证明和论述,因为只有理解了的东西才能更深刻地感受它。

本书给出的例子,总量达到近250个,其中很多例子更贴近人们的社会、经济、生活和生产管理,更具有时代气息。这些例子是我们日常教学和研究中收集起来的,它能把概率统计基本内容渗透到各种实际中去。

本书的习题分节设立,这样可使习题更具针对性,并通过习题增强能力和扩大视野。习题数量也明显增多,全书有600道习题。这些习题中一半左右是基
本题,使大多数学生在掌握基本知识后都能做出,还有一部分习题经过努力大多也能做出,这样安排习题是希望培养学生兴趣与能力,提高学生学好这门课程的信心。另外,配合本书的教与学,我们还编了一本“概率论与数理统计习题与解答”,将于近期出版。这本辅助读物有助于把学生的兴趣和能力引向更深的层次,亦起到“解惑”的作用。

使用本书有两个建议方案,若概率论与数理统计分两学期开设,每学期60学时,本书可在120学时左右全部讲完。这正是本书编写的初衷。若概率论与数理统计作为一门课程在一学期开设,可选择部分内容组织教学,性如,

\begin{itemize}
  \item 概率论部分可选第一、二章大部分内容加上数学期望与方差运算性质、伯努利大数定律和中心极限定理。
  \item 统计部分可选第五、六、七章大部分内容,其中充分统计量、最小方差无偏估计、两样本的假设检验均可略去。
\end{itemize}

在此我们首先感谢华东师范大学统计系领导和全体教师,由于他(她)们的关心、支持和鼓励使我们能以充沛的糖力去完成此书。我们还要感谢葛广乎教授,他在百忙之中审阅了全部书稿,提出了宝贵意见。由于采纳了他的改进意见,使本书的质量进一步得到了提高。最后要感谢高等教育出版扯理科分扯对本书的支持和督促,没有他(她)们的热心指导和出色编辑,不可能使本书迅速问世。

本书前四章由程依明编写,后四章由濮晚龙编写,全书由弗诗松统稿。我们经常讨论、切碰写法、选择例题、相互补充,终于完成此书。由于水乎有限,不当之处在所难免,愿请广大教师和学生提出宝贵意见,我们将作进一步改进。

\hfill 莽诗松、程依明、濮晚龙

\hfill 2004年3月 \hspace{0.8cm}

\chapter*{前言}
\addcontentsline{toc}{chapter}{前言}
本书的重排项目本来是由github的re-book项目组于2019年4月份发起, 原本是每人一章分工合作,但是中间陆续有人退出,大量排版工作没有完成,还有巨多的图要画,我完成了分配给自己的第五章,但是其他很多的章节落下了太多,后来就干脆变成了一个烂尾的项目。如今时隔两年,我觉得这本书还是有完成的必要,于是自己重操旧业,准备来完成这本书的剩余部分,然后制作附录表格,并将所有的图重新用TikZ绘制,相信这本书将来会对很多人的学习产生帮助。

由于这本书是分工完成的,不同人代码风格差异很大,文字是用ORC识别的,也有所缺失。很多公式都是机器代码,我也没有精力再去一一修改。如果读者在阅读过程中发现了排版错误的,欢迎致邮勘误. \href{739049687@qq.com}{739049687@qq.com}

\hfill 向禹\hspace{2em}

\hfill 2020年3月

% !TeX program = XeLaTeX
% !TeX root = main.tex
% Edit by:
\chapter{方差分析与回归分析\label{cha:8}}

\section{方差分析\label{sec:8.1}}
\subsection{问题的提出}
前面几章我们讨论的都是一个总体或者两个总体的统计分析问题,在实际工作中我们还会经常碰到多个总体均值的比较问题,处理这类问题通常采用所谓的方差分析方法。本节将叙述这个方法,先看一个例子。

\begin{example}\label{exam:8.1.1}
在饲料养鸡增肥的研究中,某研究所提出三种饲料配方:$A_1$ 是以鱼粉为主的饲料,$A_2$ 是以槐树粉为主的饲料,$A_3$ 是以苜蓿粉为主的饲料。为比较三种饲料的效果,特选 24 只相似的雏鸡随机均分为三组,每组各喂一种饲料,60 天后观察它们的重量。试验结果如下表所示:

\begin{table}[htbp]
  \centering
  \caption{鸡饲料试验数据}
    \begin{tabular}{c|rrrrrrrr}
    \toprule
    饲料 $A$   & \multicolumn{7}{c}{鸡重/\si{\gram}                     } &      \\
    \midrule
    $A_1$  & 1073  & 1009  & 1060  & 1001  & 1002  & 1012  & 1009  & 1028 \\
    $A_2$  & 1107  & 1092  & 990   & 1109  & 1090  & 1074  & 1122  & 1001 \\
    $A_3$  & 1093  & 1029  & 1080  & 1021  & 1022  & 1032  & 1029  & 1048 \\
    \bottomrule
    \end{tabular}%
  \label{tab:8.1.1}%
\end{table}%
\end{example}

本例中,我们要比较的是三种饲料对鸡的增肥作用是否相同。为此,把饲料称为 \index{因子},记为 $A$,三种不同的配方称为因子 $A$ 的三个水平,记为 $A_1$,$A_2$,$A_3$,使用配方 $A_i$ 下第 $j$ 只鸡 60 天后的重量用 $y_{ij}$ 表示,$i = 1,2,3$,$j =1,2,3,\ldots,10$。我们的目的是比较三种不同饲料配方下鸡的平均重量是否相等,为此,需要做一些基本假定,把所研究的问题归结为一个统计问题,然后用方差分析的方法进行解决。

在例~\ref{exam:8.1.1} 中,我们只考察了一个因子,称其为单因子试验。通常,在单因子试验中,记因子为 $A$,设其有 $r$ 个水平,记为 $A_1,A_2,\ldots,A_r$,在每一水平下考察的指标可以看成一个总体,现有 $r$ 个水平,故有 $r$ 个总体,假定:

\begin{enumerate}
  \item 每一总体均为正态分布,记为 $N(\mu_i, \sigma_i^2)$,$i=1,\ldots,r$;\label{enu:8.1.2.1}
  \item 各总体的方差相同,记为 $\sigma_1^2 = \sigma_2^2=\cdots=\sigma_r^2=\sigma^2$;\label{enu:8.1.2.2}
  \item 从每一总体中抽取的样本是相互独立的,即所有的试验结果 $y_{ij}$ 都相互独立。
\end{enumerate}

这三个假定都可以用统计方法进行验证。譬如,利用正态性检验(7.4.3 节)验证~\ref{enu:8.1.2.1} 成立;利用后面~\ref{sec:8.3} 的方差齐次性检验验证~\ref{enu:8.1.2.2} 成立;而试验结果 $y_{ij}$ 的独立性可由随机化实现,这里的随机化是指所有试验按随机次序进行。

我们要做的工作是比较各水平下的均值是否相同,即要对如下的一个假设进行检验,

\begin{equation}
  H_0 \textrm{ : } \mu_1 = \mu_2 = \cdots = \mu_r,
\end{equation}



\section{多重比较\label{sec:8.2}}

\section{方差齐次检验\label{sec:8.3}}

\section{一元线性回归\label{sec:8.4}}

\section{一元非线性回归\label{sec:8.5}}
% !TeX program = XeLaTeX
% !TeX root = main.tex
% Edit by:
\chapter{方差分析与回归分析}

\section{方差分析}
\subsection{问题的提出}
前面几章我们讨论的都是一个总体或者两个总体的统计分析问题,在实际工作中我们还会经常碰到多个总体均值的比较问题,处理这类问题通常采用所谓的方差分析方法。本节将叙述这个方法,先看一个例子。

\begin{example}
在饲料养鸡增肥的研究中,某研究所提出三种饲料配方:$A_1$ 是以鱼粉为主的饲料,$A_2$ 是以槐树粉为主的饲料,$A_3$ 是以苜蓿粉为主的饲料。为比较三种饲料的效果,特选 24 只相似的雏鸡随机均分为三组,每组各喂一种饲料,60 天后观察它们的重量。试验结果如下表所示:

\begin{table}[htbp]
  \centering
  \caption{鸡饲料试验数据}
    \begin{tabular}{c|rrrrrrrr}
    \toprule
    饲料 $A$   & \multicolumn{7}{c}{鸡重/\si{\gram}                     } &      \\
    \midrule
    $A_1$  & 1073  & 1009  & 1060  & 1001  & 1002  & 1012  & 1009  & 1028 \\
    $A_2$  & 1107  & 1092  & 990   & 1109  & 1090  & 1074  & 1122  & 1001 \\
    $A_3$  & 1093  & 1029  & 1080  & 1021  & 1022  & 1032  & 1029  & 1048 \\
    \bottomrule
    \end{tabular}%
  \label{tab:8.1.1}%
\end{table}%
\end{example}

本例中,我们要比较的是三种饲料对鸡的增肥作用是否相同。为此,把饲料称为 \index{因子},记为 $A$,三种不同的配方称为因子 $A$ 的三个水平,记为 $A_1$,$A_2$,$A_3$,使用配方 $A_i$ 下第 $j$ 只鸡 60 天后的重量用 $y_{ij}$ 表示,$i = 1,2,3$,$j =1,2,3,\ldots,10$。我们的目的是比较三种不同饲料配方下鸡的平均重量是否相等,为此,需要做一些基本假定,把所研究的问题归结为一个统计问题,然后用方差分析的方法进行解决。


% !TeX program = XeLaTeX
% !TeX root = main.tex
% Edit by:
\chapter{方差分析与回归分析}\label{cha:8}

\section{方差分析}\label{sec:8.1}
\subsection{问题的提出}
前面几章我们讨论的都是一个总体或者两个总体的统计分析问题, 在实际工作中我们还会经常碰到多个总体均值的比较问题, 处理这类问题通常采用所谓的方差分析方法. 本节将叙述这个方法, 先看一个例子. 

\begin{example}\label{exam:8.1.1}
在饲料养鸡增肥的研究中, 某研究所提出三种饲料配方:$A_1$ 是以鱼粉为主的饲料, $A_2$ 是以槐树粉为主的饲料, $A_3$ 是以苜蓿粉为主的饲料. 为比较三种饲料的效果, 特选 24 只相似的雏鸡随机均分为三组, 每组各喂一种饲料, 60 天后观察它们的重量. 试验结果如下表所示:

\begin{table}[htbp]
  \centering
  \caption{鸡饲料试验数据}
    \begin{tabular}{c|rrrrrrrr}
    \toprule
    饲料 $A$   & \multicolumn{7}{c}{鸡重/\si{\gram}                     } &      \\
    \midrule
    $A_1$  & 1073  & 1009  & 1060  & 1001  & 1002  & 1012  & 1009  & 1028 \\
    $A_2$  & 1107  & 1092  & 990   & 1109  & 1090  & 1074  & 1122  & 1001 \\
    $A_3$  & 1093  & 1029  & 1080  & 1021  & 1022  & 1032  & 1029  & 1048 \\
    \bottomrule
    \end{tabular}%
  \label{tab:8.1.1}%
\end{table}%
\end{example}

本例中, 我们要比较的是三种饲料对鸡的增肥作用是否相同. 为此, 把饲料称为\textbf{因子}\index{Y!因子}, 记为 $A$, 三种不同的配方称为因子 $A$ 的三个水平, 记为 $A_1$, $A_2$, $A_3$, 使用配方 $A_i$ 下第 $j$ 只鸡 60 天后的重量用 $y_{ij}$ 表示, $i = 1,2,3$, $j =1,2,3,\ldots,10$. 我们的目的是比较三种不同饲料配方下鸡的平均重量是否相等, 为此, 需要做一些基本假定, 把所研究的问题归结为一个统计问题, 然后用方差分析的方法进行解决. 

在例~\ref{exam:8.1.1} 中, 我们只考察了一个因子, 称其为单因子试验. 通常, 在单因子试验中, 记因子为 $A$, 设其有 $r$ 个水平, 记为 $A_1,A_2,\ldots,A_r$, 在每一水平下考察的指标可以看成一个总体, 现有 $r$ 个水平, 故有 $r$ 个总体, 假定:

\begin{enumerate}
  \item 每一总体均为正态分布, 记为 $N(\mu_i, \sigma_i^2)$, $i=1,\ldots,r$;\label{enu:8.1.2.1}
  \item 各总体的方差相同, 记为 $\sigma_1^2 = \sigma_2^2=\cdots=\sigma_r^2=\sigma^2$;\label{enu:8.1.2.2}
  \item 从每一总体中抽取的样本是相互独立的, 即所有的试验结果 $y_{ij}$ 都相互独立. 
\end{enumerate}

这三个假定都可以用统计方法进行验证. 譬如, 利用正态性检验(7.4.3 节)验证~\ref{enu:8.1.2.1} 成立;利用后面~\ref{sec:8.3} 的方差齐次性检验验证~\ref{enu:8.1.2.2} 成立;而试验结果 $y_{ij}$ 的独立性可由随机化实现, 这里的随机化是指所有试验按随机次序进行. 

我们要做的工作是比较各水平下的均值是否相同, 即要对如下的一个假设进行检验, 
\begin{equation}
  H_0 \text{:} \mu_1 = \mu_2 = \cdots = \mu_r,\label{eq:8.1.1}
\end{equation}
其备择假设为
\begin{equation*}
  H_1 \text{ : } \mu_1,\mu_2,\ldots,\mu_r \text{ 不全相等, }
\end{equation*}
在不会引起误解的情况下, $H_1$ 通常可省略不写. 

如果 $H_0$ 成立, 因子 $A$ 的 $r$ 个水平均值相同, 称因子 $A$ 的 $r$ 个水平间没有显著差异, 简称因子 $A$ \textbf{不显著}\index{B!不显著};反之, 当 $H_0$ 不成立时, 因子 $A$ 的 $r$ 个水平均值不全相同, 这时称因子 $A$ 的不同水平间有显著差异, 简称因子 $A$ \textbf{显著}\index{X!显著}. 

为对假设~(\ref{eq:8.1.1}) 进行检验, 需要从每一水平下的总体抽取样本, 设从第 $i$ 个水平下的总体获得 $m$ 个试验结果(简单起见, 这里先假设个水平下试验的重复数相同, 后面会看到, 重复数不同时的处理方式与此基本一致, 略有差异),记 $y_{ij}$ 表示第 $i$ 个总体的第 $j$ 次重复试验结果. 共得到如下 $r \times m$ 个试验结果:
\begin{equation*}
  y_{ij}, \; i=1,2,\ldots,r, \; j = 1,2,\ldots,m,
\end{equation*}
其中 $r$ 为水平数, $m$ 为重复数, $i$ 为水平编号, $j$ 为重复编号.

在水平 $A_i$ 下的试验结果 $y_{ij}$ 与该水平下的指标均值 $\mu_i$ 一般总是有差距的, 记 $\varepsilon_{ij} = y_{ij} - \mu_i$, $\varepsilon_{ij}$ 称为随机误差. 于是有
\begin{equation}
  \label{eq:8.1.2}
  y_{ij} = \mu_i + \varepsilon_{ij}
\end{equation}

~(\ref{eq:8.1.2}) 式称为试验结果 $y_{ij}$ 的\textbf{数据结构式}\index{S!数据结构式}. 把三个假定用子数据结构式就可以写出单因子方差分析的统计模型:
\begin{equation}
  \label{eq:8.1.3}
  \begin{cases}
    y_{ij}  = \mu_i + \varepsilon_{ij}, \; i = 1,2,\ldots,r,\; j = 1,2,\ldots,m; \\
    \text{诸 $\varepsilon_{ij}$ 相互独立, 且都服从 $N(0,\sigma^2)$ }.
  \end{cases}
\end{equation}

为了能更好地描述数据, 常在方差分析中引入总均值与效应的概念. 称诸 $\mu_i$ 的平均(所有试验结果的均值的平均)
\begin{equation}
  \label{eq.8.1.4}
  \mu = \frac{1}{r} (\mu_1 + \cdots + \mu_r) = \frac{1}{r} \sum_{i=1}^{r} \mu_i
\end{equation}
为总均值. 称第 $i$ 水平下的均值 $\mu_i$ 与总均值 $\mu$ 的差
\begin{equation}
  \label{eq:8.1.5}
  a_i = \mu_i - \mu, \quad i = 1,2,\ldots,r
\end{equation}
为因子 $A$ 的第 $i$ 水平的\textbf{主效应}\index{Z!主效应}, 简称为 $A_i$ 的效应. 

容易看出 
% \begin{align}
%   \sum_{i=1}^{r} a_i = 0, \label{eq:8.1.6}\\
%   \mu_i = \mu + a_i, \label{eq:8.1.7}
% \end{align}
\begin{equation}
  \label{eq:8.1.6}
  \sum_{i=1}^{r} a_i = 0,
\end{equation}
\begin{equation}
  \label{eq:8.1.7}
  \mu_i = \mu + a_i,
\end{equation}
这表明第 $i$ 个总体均值是由总均值与该水平的效应叠加而成的, 从而模型~(\ref{eq:8.1.3}) 可以改写为
\begin{equation}
  \label{eq:8.1.8}
  \begin{cases}
    y_{ij}  = \mu + a_i + \varepsilon_{ij}, \quad i = 1,2,\ldots,r,\; j = 1,2,\ldots,m; \\
    \sum\limits_{i=1}^{r} a_i = 0; \\
    \text{$\varepsilon_{ij}$ 相互独立, 且都服从 $N(0,\sigma^2)$ }.
  \end{cases}
\end{equation}
假设~(\ref{eq:8.1.1}) 可改写为 
\begin{equation}
  H_0 \textrm{ : } a_1 = a_2 = \cdots = a_r,\label{eq:8.1.9}
\end{equation}
其备择假设为
\begin{equation*}
  H_1 \text{ : } a_1,a_2,\ldots,a_r \text{ 不全为 0.}
\end{equation*}

\subsection{平方和分解}\label{ssec:8.1.3}
\textbf{一、试验数据}

通常在单因子方差分析中可将试验数据列成如下表格形式. 

% Table generated by Excel2LaTeX from sheet 'Sheet1'
\begin{table}[htbp]
  \centering
  \caption{单因子方差分析试验数据}
    \begin{tabular}{ccccccccc}
    \toprule
    因子水平  &       & \multicolumn{4}{c}{试验数据}      &       & 和     & 平均 \\
    \midrule
    $A_1$    &       & $y_{11}$   & $y_{12}$   & $\cdots$ & $y_{1m}$   &       & $T_1$    & $\bar{y}_{1}$ \\
    $A_2$    &       & $y_{21}$   & $y_{22}$   & $\cdots$  & $y_{2m}$   &       & $T_2$    & $\bar{y}_{2}$ \\
    $\vdots$ &       & $\vdots$    & $\vdots$   &       &   $\vdots$  &       &  $\vdots$   & $\vdots$ \\
    $A_r$    &       & $y_{r1}$   & $y_{r2}$   & $\cdots$   & $y_{rm}$   &       & $T_r$    & $\bar{y}_{r}$ \\
    \midrule
          &       &       &       &       &       &       & $T$     & $\bar{y}$ \\
    \bottomrule
    \end{tabular}%
  \label{tab:8.1.2}%
\end{table}%
\ref{tab:8.1.2} 中的最后二列的和与平均的含义如下:
\begin{align*}
  T_i &= \sum_{j=1}^{m} y_{ij},\; \bar{y}_i = \frac{T_i}{m} \quad i =1,2,\ldots,r,\\
  T_i & = \sum_{i=1}^{r} T_{i},\; \bar{y} = \frac{T}{r \cdot m} = \frac{T}{n},\\
  n & = r \cdot m = \text{ 总试验次数 }.
\end{align*}

\textbf{二、组内偏差与组间偏差}
数据间是有差异的. 数据 $y_{ij}$ 与总平均 $\bar{y}$ 间的偏差可用 $y_{ij} - \bar{y}$ 表示, 它可分解为两个偏差之和
\begin{equation}\label{eq:8.1.10}
  y_{ij}-\bar{y}=(y_{ij}-\bar{y}_{i.})+(\bar{y}_{i.}-\bar{y})
\end{equation}
记
\begin{equation*}
  \bar{\varepsilon}_{i.} =\frac{1}{m} \sum_{j=1}^{m} \varepsilon_{ij}, \quad \bar{\varepsilon}=\frac{1}{r} \sum_{i=1}^{r} \bar{\varepsilon}_{i} = \frac{1}{n} \sum_{i=1}^{r} \sum_{j=1}^{m} \varepsilon_{ij}.
\end{equation*}
由于
\begin{equation}\label{eq:8.1.11}
  y_{i j}-\bar{y}_{i.}=(\mu_{i}+\varepsilon_{i j})-(\mu_{i}+\bar{\varepsilon}_{i})=\varepsilon_{i j}-\bar{\varepsilon}_{i},
\end{equation}
所以 $y_{ij} - \bar{y}_{i.}$ 仅反映组内数据与组内平均的随机误差, 称为\textbf{组内偏差}\index{Z!组内偏差}; 而
\begin{equation}\label{eq:8.1.12}
  \bar{y}_{i.}-\bar{y} = (\mu_{i}+ \bar{\varepsilon}_{i.})-(\mu +\bar{\varepsilon}_{i})= a_i + \bar{\varepsilon}_{i.}-\bar{\varepsilon},
\end{equation}
$\bar{y}_{i.} - \bar{y}$ 除了反映随机误差外,还反映了第 $i$ 个水平的效应,称为组间偏差,

\textbf{三、偏差平方和及其自由度}

在统计学中,把 $k$ 个数据 $y_1,\ldots,y_k$ 分别对其均值 $\bar{y}=(y_1+\cdots+y_{k}) / k$ 的偏差平方和
\begin{equation*}
  Q=\left(y_{1}-\bar{y}\right)^{2}+\cdots+\left(y_{k}-\bar{y}\right)^{2}=\sum_{i=1}^{k}\left(y_{i}-\bar{y}\right)^{2}
\end{equation*}

称为 $k$ 个数据的\textbf{偏差平方和}\index{P!偏差平方和},有时简称\textbf{平方和}\index{P:平方和}. 偏差平方和常用来度量若干个数据集中或分散的程度,它是用来度量若干个数据间差异(即波动)的大小的一个重要的统计量.

在构成偏差平方和 $Q$ 的 $k$ 个偏差 $y_1 - \bar{y}, \ldots, y_{k} - \bar{y}$ 间有一个恒等式
\begin{equation*}
  \sum_{i=1}^{k}\left(y_{i}-\bar{y}\right)=0
\end{equation*}

这说明在 $Q$ 中独立的偏差只有 $k-1$ 个. 在统计学中把平方和中独立偏差个数称为该平方和的\textbf{自由度}\index{Z:自由度},常记为 $f$ ,如 $Q$ 的自由度为 $f_{Q}=k-1$. 自由度是偏差平方和的一个重要参数.

\textbf{四、总平方和分解公式}

各 $y_{ij}$ 间总的差异大小可用\textbf{总偏差平方和} $S_T$ 表示,

\begin{equation}\label{eq:8.1.13}
  S_{T}=\sum_{i=1}^{r} \sum_{j=1}^{m}\left(y_{i j}-\bar{y}\right)^{2}, \quad f_{T} = n-1,
\end{equation}

仅由随机误差引起的数据间的差异可以用组内偏差平方和表示,也称为误差偏差平方和,记为 $S_e$
\begin{equation}\label{eq:8.1.14}
  S_{e}=\sum_{i=1}^{r} \sum_{j=1}^{m}\left(y_{i j}-\bar{y}_{i .}\right)^{2}, \quad f_{e}=r(m-1)=n-r.
\end{equation}
由于组间差异除了随机误差外,还反映了效应间的差异,故由效应不同引起的数据差异可用\textbf{组间偏差平方和}表示,也称为因子 $A$ 的\textbf{偏差平方和},记为 $S_A$;
\begin{equation}\label{eq:8.1.15}
  S_{A}=m \sum_{i=1}^{r}\left(\bar{y}_{i.}-\bar{y}\right)^{2}, \quad f_{A}=r-1
\end{equation}

\begin{theorem}{}{8.1.1}
在上述符号下,总平方和 $S_T$ 可以分解为因子平方和 $S_A$ 与误差平方和 $S_e$。之和,其自由度也有相应分解公式,具体为:
\begin{equation} \label{eq:8.1.16}
S_{T}=S_{A}+ S_{e}, \quad f_{T}=f_{A}+f_{e}
\end{equation}
~\eqref{eq:8.1.16} 式通常称为总平方和分解式. 
\end{theorem}

\begin{proof}
注意到
  \begin{equation}
    \sum_{i=1}^{r} \sum_{j=1}^{m}(y_{i j}-\bar{y}_{i.})(\bar{y}_{i.}-\bar{y})= \sum_{i=1}^{r}\big[(\bar{y}_{i.}-\bar{y}) \sum_{i=1}^{m}(y_{i j}-\bar{y}_{i.})\big]=0
    \end{equation}
故有
    \begin{align*}
      S_{T} & =\sum_{i=1}^{r} \sum_{j=1}^{m}(y_{i j}-\bar{y})^{2} = \sum_{i=1}^{r} \sum_{i=1}^{m}[(y_{i j}-\bar{y}_{i.})+(\bar{y}_{i.}-\bar{y})]^{2}\\
      &=S_{e}+S_{A} + 2 \sum_{i=1}^{r} \sum_{j=1}^{m}(y_{i j}-\bar{y}_{i.})+(\bar{y}_{i .}-\bar{y}) = S_{e} + S_{A},
    \end{align*}
诸自由度间的等式是显然的.
\end{proof}

\subsection{检验方法}

偏差平方和 $Q$ 的大小与数据个数(或自由度)有关,一般说来,数据越多,其偏差平方和越大. 为了便于在偏差平方和间进行比较,统计上引入了均方和的概念,它定义为 
\begin{equation*}
  MS=Q/f_Q,
\end{equation*}
其意为平均每个自由度上有多少平方和,它比较好地度量了一组数据的离散程度. 

如今要对因子平方和SA与误差平方和S2之间进行比较,用其均方和
\begin{equation*}
  MS_{A}=S_A / f_A, \quad MS_{e}=S_{e}/f_{e}
\end{equation*}


进行比较更为合理,因为均方和排除了自由度不同所产生的干扰.故用
\begin{equation}
  F = \frac{MS_{A}}{MS_{e}} = \frac{S_A/f_A}{S_e/f_e}
\end{equation}
作为检验 $H_0$ 的统计量,为给出检验拒绝域,我们需要如下定理:

\begin{theorem}{}{8.1.2}
  在单因子方差分析模型~\eqref{eq:8.1.8} 及前述符号下,有
  \begin{enumerate}
    \item $S_{\varepsilon} / \sigma^{2} \sim \chi^{2}(n-r)$, 从而 $E(S_{e})=(n-r) \sigma^{2}$ \label{enum:8.1.2.1}
    \item $E(S_{A})=(r-1) \sigma^{2} + m \sum\limits_{i=1}^{r} a_{i}^{2}$, 进一步, 若 $H_0$ 成立, 则有 $S_{A} / \sigma^{2} \sim \chi^{2}(r-1)$;\label{enum:8.1.2.2} 
    \item $S_A$ 与 $S_e$ 独立 \label{enum:8.1.2.3}.
  \end{enumerate}
\end{theorem}

\begin{proof}
由于~\eqref{eq:8.1.11} 和~\eqref{eq:8.1.14}, $S_{e}=\sum\limits_{i=1}^{r} \sum\limits_{j=1}^{m}(\varepsilon_{i j}-\bar{\varepsilon_{i.}})^{2}$, 在单因子方差分析模型~\eqref{eq:8.1.8} 下, 我们知道, 诸 $\varepsilon_{ij}, \; i=1,2,\ldots,r,\; j=1,2,\ldots,m$ 独立同分布于 $N(0,\sigma^2)$,由定理~\ref{thm:5.4.1} 知,$\tfrac{1}{\sigma^2} \sum\limits_{j=1}^m (\varepsilon_{ij} - \bar{\varepsilon}_{i.})^2$, $i=1,2,\ldots,r$, 相互独立,其共同分布为 $\chi^2(m-1)$,由卡方分布的可加性,有 $\tfrac{S_e}{\sigma^2} \sim \chi^2(n-r)$,这给出 $E(S_e/\sigma^2) = n-r=f_e$, ~\ref{enum:8.1.2.1}得证.

类似地,由~\eqref{eq:8.1.12} 和~\eqref{eq:8.1.15},有
\begin{equation*}
  S_A =  m \sum_{i=1}^r (a_i + \varepsilon_{i.} - \bar{\varepsilon})^2.
\end{equation*}
由定理~\ref{thm:5.4.1} 知,对每个 $i$,平方和 $\sum\limits_{j=1}^m (\varepsilon_{ij} - \bar{\varepsilon}_{i.})^2$ 与均值 $\bar{\varepsilon}_{i}$ 独立,从而 $\bar{\varepsilon}_{1.}, \bar{\varepsilon}_{2.},\ldots,\bar{\varepsilon}_{r.}$ 与 $S_e$ 独立,而 $S_A$ 只是,$\bar{\varepsilon}_{1.}, \bar{\varepsilon}_{2.},\ldots,\bar{\varepsilon}_{r.}$ 的函数,由此~\ref{enum:8.1.2.3} 得证.

在模型~\ref{eq:8.1.8} 下,$S_A$ 的期望是
\begin{equation*}
  E(S_A) = m \sum_{i=1}^r a^2 + E\big[m\sum_{i=1}^r (\bar{\varepsilon}_{i.} - \bar{\varepsilon})^2\big],
\end{equation*}
由于诸误差均值 $\bar{\varepsilon}_{1.}, \bar{\varepsilon}_{2.},\ldots,\bar{\varepsilon}_{r.}$ 独立同分布于 $N(0, \sigma^2/m)$,从而由诸误差均值组成的偏差平方和除以 $\sigma^2/m$ 服从卡方分布,即
\begin{equation*}
  \frac{1}{\sigma^{2}} \sum_{i=1}^{r} m\left(\bar{\varepsilon}_{i},-\bar{\varepsilon}\right)^{2}-\chi^{2}(r-1).
\end{equation*}
于是, $E\big[\sum\limits_{i=1}^r m(\bar{\varepsilon}_{i.} - \bar{\varepsilon})\big]$ 在 $H_0$ 成立下,$S_A/\sigma^2 \sim \chi^2(r-1)$, 这就完成了~\ref{enum:8.1.2.2} 的证明.
\end{proof}

\section{多重比较}\label{sec:8.2}

\section{方差齐次检验}\label{sec:8.3}

\section{一元线性回归}\label{sec:8.4}

\section{一元非线性回归}\label{sec:8.5}
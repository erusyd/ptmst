% !TeX program = XeLaTeX
% !TeX root = main.tex
% Edit by:
\chapter{方差分析与回归分析\label{cha:8}}

\section{方差分析\label{sec:8.1}}
\subsection{问题的提出}
前面几章我们讨论的都是一个总体或者两个总体的统计分析问题, 在实际工作中我们还会经常碰到多个总体均值的比较问题, 处理这类问题通常采用所谓的方差分析方法. 本节将叙述这个方法, 先看一个例子. 

\begin{example}\label{exam:8.1.1}
在饲料养鸡增肥的研究中, 某研究所提出三种饲料配方:$A_1$ 是以鱼粉为主的饲料, $A_2$ 是以槐树粉为主的饲料, $A_3$ 是以苜蓿粉为主的饲料. 为比较三种饲料的效果, 特选 24 只相似的雏鸡随机均分为三组, 每组各喂一种饲料, 60 天后观察它们的重量. 试验结果如下表所示:

\begin{table}[htbp]
  \centering
  \caption{鸡饲料试验数据}
    \begin{tabular}{c|rrrrrrrr}
    \toprule
    饲料 $A$   & \multicolumn{7}{c}{鸡重/\si{\gram}                     } &      \\
    \midrule
    $A_1$  & 1073  & 1009  & 1060  & 1001  & 1002  & 1012  & 1009  & 1028 \\
    $A_2$  & 1107  & 1092  & 990   & 1109  & 1090  & 1074  & 1122  & 1001 \\
    $A_3$  & 1093  & 1029  & 1080  & 1021  & 1022  & 1032  & 1029  & 1048 \\
    \bottomrule
    \end{tabular}%
  \label{tab:8.1.1}%
\end{table}%
\end{example}

本例中, 我们要比较的是三种饲料对鸡的增肥作用是否相同. 为此, 把饲料称为 \index{Y:因子}, 记为 $A$, 三种不同的配方称为因子 $A$ 的三个水平, 记为 $A_1$, $A_2$, $A_3$, 使用配方 $A_i$ 下第 $j$ 只鸡 60 天后的重量用 $y_{ij}$ 表示, $i = 1,2,3$, $j =1,2,3,\ldots,10$. 我们的目的是比较三种不同饲料配方下鸡的平均重量是否相等, 为此, 需要做一些基本假定, 把所研究的问题归结为一个统计问题, 然后用方差分析的方法进行解决. 

在例~\ref{exam:8.1.1} 中, 我们只考察了一个因子, 称其为单因子试验. 通常, 在单因子试验中, 记因子为 $A$, 设其有 $r$ 个水平, 记为 $A_1,A_2,\ldots,A_r$, 在每一水平下考察的指标可以看成一个总体, 现有 $r$ 个水平, 故有 $r$ 个总体, 假定:

\begin{enumerate}
  \item 每一总体均为正态分布, 记为 $N(\mu_i, \sigma_i^2)$, $i=1,\ldots,r$;\label{enu:8.1.2.1}
  \item 各总体的方差相同, 记为 $\sigma_1^2 = \sigma_2^2=\cdots=\sigma_r^2=\sigma^2$;\label{enu:8.1.2.2}
  \item 从每一总体中抽取的样本是相互独立的, 即所有的试验结果 $y_{ij}$ 都相互独立. 
\end{enumerate}

这三个假定都可以用统计方法进行验证. 譬如, 利用正态性检验(7.4.3 节)验证~\ref{enu:8.1.2.1} 成立;利用后面~\ref{sec:8.3} 的方差齐次性检验验证~\ref{enu:8.1.2.2} 成立;而试验结果 $y_{ij}$ 的独立性可由随机化实现, 这里的随机化是指所有试验按随机次序进行. 

我们要做的工作是比较各水平下的均值是否相同, 即要对如下的一个假设进行检验, 
\begin{equation}
  H_0 \text{:} \mu_1 = \mu_2 = \cdots = \mu_r,\label{eq:8.1.1}
\end{equation}
其备择假设为
\begin{equation*}
  H_1 \text{ : } \mu_1,\mu_2,\ldots,\mu_r \text{ 不全相等, }
\end{equation*}
在不会引起误解的情况下, $H_1$ 通常可省略不写. 

如果 $H_0$ 成立, 因子 $A$ 的 $r$ 个水平均值相同, 称因子 $A$ 的 $r$ 个水平间没有显著差异, 简称因子 $A$ \textbf{不显著};反之, 当 $H_0$ 不成立时, 因子 $A$ 的 $r$ 个水平均值不全相同, 这时称因子 $A$ 的不同水平间有显著差异, 简称因子 $A$ \textbf{显著}. 

为对假设~(\ref{eq:8.1.1}) 进行检验, 需要从每一水平下的总体抽取样本, 设从第 $i$ 个水平下的总体获得 $m$ 个试验结果(简单起见, 这里先假设个水平下试验的重复数相同, 后面会看到, 重复数不同时的处理方式与此基本一致, 略有差异),记 $y_{ij}$ 表示第 $i$ 个总体的第 $j$ 次重复试验结果. 共得到如下 $r \times m$ 个试验结果:
\begin{equation*}
  y_{ij}, \; i=1,2,\ldots,r, \; j = 1,2,\ldots,m,
\end{equation*}
其中 $r$ 为水平数, $m$ 为重复数, $i$ 为水平编号, $j$ 为重复编号.

在水平 $A_i$ 下的试验结果 $y_{ij}$ 与该水平下的指标均值 $\mu_i$ 一般总是有差距的, 记 $\varepsilon_{ij} = y_{ij} - \mu_i$, $\varepsilon_{ij}$ 称为随机误差. 于是有
\begin{equation}
  \label{eq:8.1.2}
  y_{ij} = \mu_i + \varepsilon_{ij}
\end{equation}

~(\ref{eq:8.1.2}) 式称为试验结果 $y_{ij}$ 的\textbf{数据结构式}. 把三个假定用子数据结构式就可以写出单因子方差分析的统计模型:
\begin{equation}
  \label{eq:8.1.3}
  \begin{cases}
    y_{ij}  = \mu_i + \varepsilon_{ij}, \; i = 1,2,\ldots,r,\; j = 1,2,\ldots,m; \\
    \text{诸 $\varepsilon_{ij}$ 相互独立, 且都服从 $N(0,\sigma^2)$ }.
  \end{cases}
\end{equation}

为了能更好地描述数据, 常在方差分析中引入总均值与效应的概念. 称诸 $\mu_i$ 的平均(所有试验结果的均值的平均)
\begin{equation}
  \label{eq.8.1.4}
  \mu = \frac{1}{r} (\mu_1 + \cdots + \mu_r) = \frac{1}{r} \sum_{i=1}^{r} \mu_i
\end{equation}
为总均值. 称第 $i$ 水平下的均值 $\mu_i$ 与总均值 $\mu$ 的差
\begin{equation}
  \label{eq:8.1.5}
  a_i = \mu_i - \mu, \quad i = 1,2,\ldots,r
\end{equation}
为因子 $A$ 的第 $i$ 水平的\index{Z:主效应}, 简称为 $A_i$ 的效应. 

容易看出 
\begin{align}
  \sum_{i=1}^{r} a_i = 0, \label{eq:8.1.6}\\
  \mu_i = \mu + a_i, \label{eq:8.1.7}
\end{align}
这表明第 $i$ 个总体均值是由总均值与该水平的效应叠加而成的, 从而模型~(\ref{eq:8.1.3}) 可以改写为
\begin{equation}
  \label{eq:8.1.8}
  \begin{cases}
    y_{ij}  = \mu + a_i + \varepsilon_{ij}, \quad i = 1,2,\ldots,r,\; j = 1,2,\ldots,m; \\
    \sum\limits_{i=1}^{r} a_i = 0; \\
    \text{$\varepsilon_{ij}$ 相互独立, 且都服从 $N(0,\sigma^2)$ }.
  \end{cases}
\end{equation}
假设~(\ref{eq:8.1.1}) 可改写为 
\begin{equation}
  H_0 \textrm{ : } a_1 = a_2 = \cdots = a_r,\label{eq:8.1.9}
\end{equation}
其备择假设为
\begin{equation*}
  H_1 \text{ : } a_1,a_2,\ldots,a_r \text{ 不全为 0.}
\end{equation*}

\subsection{平方和分解\label{ssec:8.1.3}}
\textbf{一、试验数据}

通常在单因子方差分析中可将试验数据列成如下表格形式. 

% Table generated by Excel2LaTeX from sheet 'Sheet1'
\begin{table}[htbp]
  \centering
  \caption{单因子方差分析试验数据}
    \begin{tabular}{ccccccccc}
    \toprule
    因子水平  &       & \multicolumn{4}{c}{试验数据}      &       & 和     & 平均 \\
    \midrule
    $A_1$    &       & $y_{11}$   & $y_{12}$   & $\cdots$ & $y_{1m}$   &       & $T_1$    & $\bar{y}_{1}$ \\
    $A_2$    &       & $y_{21}$   & $y_{22}$   & $\cdots$  & $y_{2m}$   &       & $T_2$    & $\bar{y}_{2}$ \\
    $\vdots$ &       & $\vdots$    & $\vdots$   &       &   $\vdots$  &       &  $\vdots$   & $\vdots$ \\
    $A_r$    &       & $y_{r1}$   & $y_{r2}$   & $\cdots$   & $y_{rm}$   &       & $T_r$    & $\bar{y}_{r}$ \\
    \midrule
          &       &       &       &       &       &       & $T$     & $\bar{y}$ \\
    \bottomrule
    \end{tabular}%
  \label{tab:8.1.2}%
\end{table}%
\ref{tab:8.1.2} 中的最后二列的和与平均的含义如下:
\begin{align*}
  T_i &= \sum_{j=1}^{m} y_{ij},\; \bar{y}_i = \frac{T_i}{m} \quad i =1,2,\ldots,r,\\
  T_i & = \sum_{i=1}^{r} T_{i},\; \bar{y} = \frac{T}{r \cdot m} = \frac{T}{n},\\
  n & = r \cdot m = \text{ 总试验次数 }.
\end{align*}

\section{多重比较\label{sec:8.2}}

\section{方差齐次检验\label{sec:8.3}}

\section{一元线性回归\label{sec:8.4}}

\section{一元非线性回归\label{sec:8.5}}
% Edit by: 八一
\chapter{假设检验\label{cha:7}}
统计推断的另一个主要内容是统计假设检验。在这一章里我们将讨论统计假设的设立及其检验问题。

\section{假设检验的基本思想与概念}\label{sec:7.1}
\subsection{假设检验问题\label{7.1.1}}
我们从一个例子开始引出假设检验问题。
\begin{problem}\label{exam7.0.1}
某厂生产的合金强度服从正态分布 $N(\theta,16)$ ,其中 $\theta$ 的设计值为不低于 110Pa .为保证质量,该厂每天都要对生产情况做例行检查,以判断生产是否正常进行,即该合金的平均强度不低于 110(pa) .某天从生产中随机抽取25块合金,测得强度值为 $x_{1},x_{2},\dotsc,x_{25}$ 其均值为 $\overline{x}=108(\mathrm {Pa})$ ,间当日生产是否正常?

对这个实际问题可作如下分析;
\begin{enumerate}
	\item 这不是一个参数估计问题.
	
	\item 这是在给定总体与样本下,要求对命题“合金平均强度不低于110Pa”作出回答:“是”还是“否”?这类问题称为统计假设检验问题,简称假设检验问题.
	
	\item 命题:“合金平均强度不低于110Pa”正确与否仅涉及参数0,因此该命题是否正确将涉及如下两个参数集合:
	\[
	\theta_0=\left|\theta ;\theta\geq 110\right|,\\\theta_1=\left|\theta :\theta <110\right|
	\]
	命题成立对应于 “$\theta\in\theta_0$” ,命题不成立则对应 “$\theta\in\theta_1$” .在统计学中这两个非空参数集合都称作统计假设,简称假设.
	\item 我们的任务是利用所给总体 $N(\theta,16)$ 和样本均值 $\overline{x}=108(\mathrm {Pa})$去判断假设(命题) “$\theta\in\theta_0$” .是否成立,这里的“判断”在统计学中称为检验或检验法则。
	
	检验结果有两种:
	\begin{center}
		“假设不正确”——称为拒绝该假设;
		
		“假设正确”——称为接收该假设.
	\end{center}
	\item 若假设可用一个参数的集合表示,该假设检验问题称为参数假设检验问题,否则称为非参数假设检验问题,例~\ref{exam7.0.1}就是一个参数假设检验问题,而对假设“总体为正态分布”作出检验的问题就是一个非参数假设检验问题.本章前三节讲述参数假设检验问题,最后一节(7.4)将讨论非参数假设检验问题.
\end{enumerate}

\end{problem}

\subsection{假设检验的基本步骤\label{7.1.2}}
接下来我们来叙述假设检验的基本步骤.

\begin{xiti}
	\item 设$x_1,\dotsc ,x_n$是来自 $N(\mu ,1)$ 的样本,考虑如下假设检验问题
	\[H _ { 0 } : \mu = 2 \quad \text { vs } \quad H _ { 1 } , \mu = 3\]
	\begin{enumerate}
		\item 当 $n=20$ 时求检验犯两类错误的概率;
		\item 如果要使得检验犯第二类错误的概率$\beta \leq 0.01$,$n$最小应取多少?
		
		\item 证明:当 $n\rightarrow +\infty $时,$\alpha \rightarrow  0$,$\beta \rightarrow  0$.
	\end{enumerate}
	\item 设$x_1,\dotsc ,x_{10}$是来自 $0-1$ 总体 $b(1,p)$ 的样本,考虑如下假设检验问题
	\[H _ { 0 } : p = 0.2 \quad \text { vs } \quad H _ { 1 } , p = 0.4\]
	取拒绝域为$W = \{ \overline { x } \geq 0.5 |$,求该检验犯两类错误的概率.
\end{xiti}


\section{正态总体参数假设检验}\label{sec:7.2}
参数假设检验常见的有三种基本形式

\[\begin{array} { l } { \text { (1) } H _ { 0 } : \theta \leq \theta _ { 0 } \quad \text { vs } H _ { 1 } : \theta > \theta _ { 0 } } \\ { \text { (2) } H _ { 0 } : \theta \geq \theta _ { 0 } \quad \text { vs } H _ { 1 } ; \theta < \theta _ { 0 } } \\ { \text { (3) } H _ { 0 } : \theta = \theta _ { 0 } \quad \text { vs } H _ { 1 } : \theta \neq \theta _ { 0 } } \end{array}\]

一般说来,对这三种假设所采用的检验统计量是相同的,差别在拒绝域上.当备择假设 $H_{1}$ 在原假设 $H_{0}$ 一侧时的检验称为\textbf{单侧检验}\index{Y!单侧检验},当备择假设 $H_{1}$ 分散在原假设 $H_{0}$ 两侧时的检验称为\textbf{双侧检验}\index{Y!双侧检验}.以上(1),(2)是单侧检验,(3)是双侧检验.识别单侧与双侧检验有益于以后构造其拒绝域.

本节对正态总体参数检验分别进行讨论.
\subsection{单个正态总体均值的检验\label{7.2.1}}

\subsection{两个正态总体均值差的检验\label{7.2.2}}


\subsection{正态总体方整的检验\label{7.2.3}}

\begin{xiti}
	\item 有一批枪弹,出厂时,其初速 $v~N(950,100)$(单位:m/s).经过较长时间储存,取9发进行测试,得样本值(单位:m/s)如下:
	\[914 \quad 920 \quad 910 \quad 934 \quad 953 \quad 945 \quad 912 \quad 924 \quad 940\]
	据经验,枪弹经储存后其初速仍服从正态分布,且标准差保持不变,问是否可认为这批枪弹的初速有显著降低 $(\alpha=0.05)$?
		
	\item 已知某炼铁厂铁水含碳量服从正态分布 $N(4.55,0.108^{{2}})$.现在测定了9炉铁水,其平均含碳量为4.484,如果铁水含碳量的方差没有变化,可否认为现在生产的铁水平均含碳量仍为4.55 $(\alpha=0.05)$?
\end{xiti}


\section{其他分布参数的假设检验}\label{sec:7.3}
\subsection{指数分布参数的假设检验}\label{sec:7.3.1}

\subsection{比例 $p$ 检验}\label{sec:7.3.2}
\subsection{大样本检验}\label{sec:7.3.3}

\subsection{检验的 $p$ 值}\label{sec:7.3.4}

\begin{xiti}
	\item 从一批服从指数分布的产品中抽取10个进行寿命试验,观测值如下(单位:h):
	\[1643 \quad 1629 \quad 426 \quad 132 \quad 1522 \quad 432 \quad 1759 \quad 1074 \quad 528 \quad 283\]
	根据这批数据能否认为其平均寿命不低于1100h?(取$(\alpha=0.05)$
		
	\item 某厂一种元件平均使用寿命为1200h,偏低,现厂里进行技术革新,革新后任选8个元件进行寿命试验,测得寿命数据如下:
	\[2686 \quad 2001 \quad 2082 \quad 792 \quad 1660 \quad 4105 \quad 1416 \quad 2089\]
	假定元件寿命服从指数分布,取 $(\alpha=0.05)$,问革新后元件的平均寿命是否有明显提高?
\end{xiti}
\section{分布拟合检验}\label{sec:7.4}
在前面我们讨论的检验问题都是在总体分布形式已知的前提下对分布的参数建立假设并进行检验,它们都属于参数假设检验问题.下面我们对总体分布的形式建立假设并进行检验.这一类检验问题统称为分布的拟合检验,它们是一类非参数检验问题.

\subsection{总体分布只取有限个值的情况}\label{sec:7.3.1}

\subsection{列联表的独立性检验}\label{sec:7.3.2}
\subsection{正态性检验}\label{sec:7.3.3}

\begin{xiti}
	\item 有人对 $\pi =3.1415926\dotsc $的小数点后800位数字中数字 $0,1,2,\dotsc,9$ 出现的次数进行了统计,结果如下
	
	\begin{table}[!htp]
		\centering
\begin{tabularx}{1.0\textwidth}{Z|ZZZZZZZZZZ}
	数字&0&1&2&3&4&5&6&7&8&9\\
			\midrule
	次数&74&92&83&79&80&73&77&75&76&91
		\end{tabularx}
	\end{table}
试在显著性水平为0.05下检验每个数字出现概率相同的假设.
\end{xiti}
% Edit by:
\chapter{假设检验}
统计推断的另一个主要内容是统计假设检验。在这一章里我们将讨论统计假设的设立及其检验问题。

\section{假设检验问题}
我们从一个例子开始引出假设检验问题。
\begin{problem}
某厂生产的合金强度服从正态分布 $N(\theta,16)$ ,其中 $\theta$ 的设计值为不低于 110Pa .为保证质量,该厂每天都要对生产情况做例行检查,以判断生产是否正常进行,即该合金的平均强度不低于 $110(pa)$ .某天从生产中随机抽取25块合金,测得强度值为 $x_{1},x_{2},\cdots,x_{25}$ 其均值为 $\overline{x}=108(\mathrm {Pa})$ ,间当日生产是否正常?

对这个实际问题可作如下分析;

(1)这不是一个参数估计问题.

(2)这是在给定总体与样本下,要求对命题“合金平均强度不低于110Pa”
作出回答:“是”还是“否”?这类问题称为统计假设检验问题,简称假设检验问题.

(3)命题:“合金平均强度不低于110Pa”正确与否仅涉及参数0,因此该命题是否正确将涉及如下两个参数集合:
\[
\theta_0=\left|\theta ;\theta\geq 110\right|,\\\theta_1=\left|\theta :\theta <110\right|
\]
命题成立对应于 $"\theta\in\theta_0"$ ,命题不成立则对应 $"\theta\in\theta_1"$ .在统计学中这两个非空参数集合都称作统计假设,简称假设.

(4)我们的任务是利用所给总体 $N(\theta,16)$ 和样本均值 $\overline{x}=108(\mathrm {Pa})$去判断假设(命题 $"\theta\in\theta_0"$ .是否成立,这里的“判断”在统计学中称为检验或检验法则。

检验结果有两种:
\begin{center}
	“假设不正确”——称为拒绝该假设;
	
	“假设正确”——称为接收该假设.
\end{center}

(5)若假设可用一个参数的集合表示,该假设检验问题称为参数假设检验问题,否则称为非参数假设检验问题,例7.1.1就是一个参数假设检验问题,而对假设“总体为正态分布”作出检验的问题就是一个非参数假设检验问题.本章前三节讲述参数假设检验问题,最后一节(7.4)将讨论非参数假设检验问题.
\end{problem}
\let\leqslant\le
\let\geqslant\ge
\let\leq\le
\let\geq\ge

\usepackage{siunitx}

\usepackage{makeidx}
\numberwithin{figure}{section}
\numberwithin{table}{section}
\numberwithin{equation}{section}

\usepackage{gbt7714}


\newcommand{\ee}{\mathrm e}
\newcommand{\dd}{\,d}

\newcommand{\textop}[1]{\relax\ifmmode\mathop{\text{#1}}\else\text{#1}\fi}
\usepackage{tabularx,booktabs}
%%%%%%定义两个列格式,数学与非数学模式
\newcolumntype{Y}{>{\centering\arraybackslash$}X<{$}}
\newcolumntype{Z}{>{\centering\arraybackslash}X}
\newcolumntype{L}{>{\raggedright\arraybackslash}X}
\newcolumntype{R}{>{\raggedleft\arraybackslash}X}
%%%%定义罗马数字
\makeatletter
\newcommand{\rmnum}[1]{\romannumeral #1}
\newcommand{\Rmnum}[1]{\expandafter\@slowromancap\romannumeral #1@}
\makeatother
\newenvironment{xiti}{\begin{center}
\textbf{\zihao{-3}\ding{45}\,习~~题\quad\thesection}
\end{center}
\begin{enumerate}[leftmargin=0.5cm]
}{\end{enumerate}} 
\allowdisplaybreaks[4]

\usepackage{mathtools}
\DeclarePairedDelimiter{\abs}{\lvert}{\rvert}

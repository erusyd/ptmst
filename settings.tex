\let\le\leqslant
\let\ge\geqslant
\let\leq\le
\let\geq\ge

\usepackage[free-standing-units]{siunitx}

\usepackage{makeidx}
\makeindex
\numberwithin{figure}{section}
\numberwithin{table}{section}
\numberwithin{equation}{section}
\usepackage{bm}

% \usepackage{gbt7714}
% \bibliographystyle{gbt7714-numerical}
\usepackage{multirow}

\newcommand{\ee}{\mathrm e}
\newcommand{\dd}{\mathop{}\!\mathrm d}

\newcommand{\textop}[1]{\relax\ifmmode\mathop{\text{#1}}\else\text{#1}\fi}
\usepackage{tabularx,booktabs,multirow}
%%%%%%定义两个列格式,数学与非数学模式
\newcolumntype{Y}{>{\centering\arraybackslash$}X<{$}}
\newcolumntype{Z}{>{\centering\arraybackslash}X}
\newcolumntype{L}{>{\raggedright\arraybackslash}X}
\newcolumntype{R}{>{\raggedleft\arraybackslash}X}

\newenvironment{xiti}{\begin{center}
\textbf{\zihao{-3}\ding{45}\,习~~题\quad\thesection}
\addcontentsline{toc}{subsection}{习题 \thesection}
\end{center}
\begin{enumerate}[leftmargin=0.5cm]
}{\end{enumerate}}

\newenvironment{answer}{\begin{center}\stepcounter{section}
\textbf{\zihao{-3}\ding{45}\,习~~题\quad\thesection}
\end{center}
\begin{enumerate}[leftmargin=0.5cm]
}{\end{enumerate}}

\usetikzlibrary{decorations.pathreplacing,decorations.pathmorphing,arrows.meta,patterns}
\newcommand{\TT}{^{\mathrm T}}
\allowdisplaybreaks[4]

\usepackage{subcaption}

% \newcommand\hmmax{0}
% \newcommand\bmmax{0}
\DeclareMathAlphabet{\mathbbm}{U}{bbm}{m}{n}

\usepackage{bm}
\DeclareMathOperator{\Cov}{Cov}
\DeclareMathOperator{\Corr}{Corr}
\DeclareMathOperator{\Var}{Var}

\let\heavymath\undefined
\let\Bbbk\undefined
\let\emptyset\varnothing
\let\dotsc\cdots
\newcommand\MR{\mathbb R}
\let\olim\lim
\def\lim{\olim\limits}
\def\Binom#1#2{
\Big(\begin{array}{@{}c@{}}
    #1\\#2
  \end{array}\Big)
}
\setmainfont{Times New Roman}


\IfFileExists{mtpro2.sty}{
  \usepackage[zswash,amsbb,straightbraces]{mtpro2}
}{\usepackage{amssymb}}

\usepackage{varwidth}
\usepackage{zhlineskip}
\usepackage[Symbol]{upgreek}
\usepackage{longtable}
\usepackage{lscape}
\usepackage{diagbox}
\let\pi\uppi
\let\vOmega\Omega
\let\Omega\varOmega
\belowrulesep0pt
\aboverulesep0pt
\setcounter{MaxMatrixCols}{25}

\edef\closure#1{%
{}\mkern2mu\overline{\mkern-2mu#1}
}
\renewcommand\bar{\closure}
\renewcommand\overline{\closure}
\newcommand\BB{\mathrm B}

\setCJKmainfont[BoldFont={FZHei-B01},ItalicFont={FZKai-Z03}]{FZShuSong-Z01}
\let\originalleft\left
\let\originalright\right
\renewcommand{\left}{\mathopen{}\mathclose\bgroup\originalleft}
\renewcommand{\right}{\aftergroup\egroup\originalright}
\newcommand\ii{\mathrm i}
\usetikzlibrary{positioning}

\makeatletter
\newcommand{\rmnum}[1]{\romannumeral #1}
\newcommand{\Rmnum}[1]{\expandafter\@slowromancap\romannumeral #1@}
\makeatother
\let\Theta\varTheta
\IfFontExistsTF{汉仪大宋简}{
  \newCJKfontfamily\hyds{汉仪大宋简}
}{
  \let\hyds\rmfamily
}
\let\bfseries\hyds
\let\Phi\varPhi
%\let\ldots\cdots

\usepackage{imakeidx}
\makeindex[
title={名词索引},
intoc=true,
columns=2,
columnsep=1cm,
columnseprule=true,
program=makeindex,
options={-s mkind.ist},
noautomatic=false
]
\indexsetup{
toclevel=chapter,
headers={名词索引}{名词索引},
othercode={
\renewcommand{\indexspace}{\smallskip}
}
}

% Edit by:曹甄强 
\chapter{随机事件与概率}
\section{随机事件及其运算}
\subsection{随机现象}

概率论与数理统计研究的对象是随机对象

在一定的条件下, 并不总是出现相同结果的现象称为随机现象, 如抛一枚硬币与掷一颗骰子.随机现象有两个特点: 

\begin{enumerate}
	\item 结果不止一个.
	\item 哪一个结果出现, 人们事先并不知道
\end{enumerate}

只有一个结果的现象称为\textbf{确定性现象}.例如, 每天早晨太阳从东方升起;水在标准大气压(~压力约为101kPa~)下加热到100$\circ$C就沸腾;一个口袋中有十只完全相同的白球, 从中任取一支必然为白球.

\begin{example}
	
	随机现象的例子
	
	\begin{enumerate}
		\item 抛一枚硬币, 有可能正面朝上, 也有可能反面朝上;
		\item 掷一颗骰子, 出现的点数;
		\item 一天内进入某超市的顾客数;
		\item 某种型号电视机的寿命;
		\item 测量某物理量(~长度、直径等~)的误差.
	\end{enumerate}
	\label{exam:1.1.1}
\end{example}

随机现象到处可见
	
在相同条件下可以重复的随机现象又称为\textbf{随机试验}.也有很多随机现象是不能重复的, 例如某场足球赛的输赢是不能重复的, 某些经济现象(~如失业、经济增长速度等~)也不能重复, 概率论与数理统计主要研究能大量重复的随机现象, 但也十分注意研究不能重复的随机现象.
	
\subsection{样本空间}
	
随机现象的一切可能基本结果组成的集合称为样本空间, 记为$\Omega=|\omega|$,其中$\omega$表示基本结果, 又称为\textbf{样本点}.样本点是今后抽样的\textbf{最基本单元}.认识随机现象首先要列出它的样本空间.
	
\begin{example}
	
	下面给出~\ref{exam:1.1.1}中随机现象的样本空间.
	\begin{enumerate}
		\item 抛一枚硬币的样本空间为: $\Omega_1=\{\omega_{1}+\omega_{2}|$, 其中$\omega_{1}\}$表示正面朝上, $\omega_{2}$表示反面朝上
		\item 掷一颗骰子的样本空间为$ \Omega_{2}=\{\omega_{1},\omega_{2},\cdots,\omega_{6}\} $, 其中$ \omega_{i} $表示出现$ i $点, $ i=1,2,\cdots,6 $. 也更直接明了地记此样本空间为: $ \Omega_{2}=\{1,2,\cdots,6\} $.
		\item 一天内进入某商场地顾客数地样本空间为: 
		\[\Omega_{3}=\{0,1,2,\cdots,500,\cdots,10^4,\cdots\},\]
		其中“0”表示“一天内无人光顾此商场”, 面“$10^4$”表示“一天内有一万人光顾此商场”.虽然此两种情况很少发生, 但我们无法说此两种情况不可能发生, 甚至于我们不能确切地说出一天内进入该商场地最多人数, 所以此样本空间用非负整数集表示,既不脱离实际情况, 又是合理抽象, 便于数学上地处理.
		\item 电视机寿命地样本空间为: $ \Omega_{4}=\{t,t \leq 0\} $.
		\item 测量误差地样本空间为:$ \Omega_{5}=\{x,-\inf < x < +\inf\} $.
	\end{enumerate}
\end{example}   

% Edit by: Xiao Xing
% !TEX root = ./main.tex

\chapter{大数定律与中心极限定理}
大数定律与中心极限定理

\section{特征函数}

设 $ p (x) $ 是随机变量 $ X $ 的密度函数,
则 $ p (x) $ 的傅里叶变换是
\begin{equation*}
    \varphi (t) = \int_{-\infty}^{+\infty} \ee^{itx} p (x) \dd x,
\end{equation*}
其中 $ i = \sqrt{-1} $ 是虚数单位.
由数学期望的概念知,
$ \varphi (t) $ 恰好是 $ E \bigl( \ee^{itx} \bigr) $.
这就是本节要讨论的特征函数,
它是处理许多概率论问题的有力工具,
它能把寻求独立随机变量和的分布的卷积运算 (积分运算) 转换成乘法运算,
还能把求分布的各阶原点矩 (积分运算) 转换成微分运算.
特别它能把寻求随机变量序列的极限分布转换成一般的函数极限问题,
下面从特征函数的定义开始介绍它们.

\subsection{特征函数的定义}

\begin{definition}{}{def:4.1.1}
    设 $ X $ 是一个随机变量,
    称
    \begin{equation}\label{eq:4.1.1}
        \varphi (t) = E \bigl( \ee^{itx} \bigr), \; -\infty \leq t \leq +\infty,
    \end{equation}
    为 $ X $ 的\textbf{特征函数}\index{特征函数}.
\end{definition}

因为 $ \lvert \ee^{itx} \rvert \leq 1 $,
所以 $ E \bigl( \ee^{itX} \bigr) $ 总是存在的,
即任一随机变量的特征函数总是存在的.

当离散随机变量 $ X $ 的分布列为 $ p_k = P ( X = x_k ) $, $ k = 1,2,\dotsc $,
则 $ X $ 的特征函数为
\begin{equation}\label{eq:4.1.2}
    \varphi (t) = \sum_{k=1}^{+\infty} \ee^{itx_k} p_k, \; -\infty \leq t \leq +\infty.
\end{equation}

当连续随机变量 $ X $ 的密度函数为 $ p (x) $,
则 $ X $ 的特征函数为
\begin{equation}\label{ch9:eq:4.1.3}
    \varphi (t) = \int_{-\infty}^{+\infty} \ee^{itx} p (x) \dd x, \; -\infty \leq t \leq +\infty.
\end{equation}

与随机变量的数学期望、方差及各阶矩一样,
特征函数只依赖于随机变量的分布,
分布相同则特征函数也相同,
所以我们也常称为某\textbf{分布的特征函数}\index{G!分布的特征函数}.

\begin{example}{常用分布的特征函数}\label{exam:4.1.1}
    \begin{enumerate}
        \item
        \textbf{单点分布}\index{单点分布}: $ P (X = a) = 1 $, 
        其特征函数为
        \begin{equation*}
            \varphi (t) = \ee^{itz}.
        \end{equation*}
        \item\label{exam:4.1.1.2}
        \textbf{0-1分布}\index{0-1分布}: $ P( X = x ) = p^x  ( 1 - p )^{1 - x} $, $ x = 0,1 $,
        其特征函数为
        \begin{equation*}
            \varphi (t) = p \ee^{it} + q, \quad \text{其中} \ q= 1 - p.
        \end{equation*}
        \item
        \textbf{泊松分布}\index{泊松分布}: $ P ( X = k ) = ( \lambda^k/k! ) \ee^{-\lambda} $, $ k = 0, 1, \dotsc $, 其特征函数为
        \begin{equation*}
            \varphi (t) = \sum_{k=0}^{+\infty} \ee^{ikt} \frac{\lambda^k}{k!} \ee^{-\lambda} = \ee^{\lambda} \ee^{\lambda \ee^{it}} = \ee^{\lambda ( \ee^{it} ) -1}.
        \end{equation*}
        \item
        \textbf{均匀分布}\index{均匀分布} $ U ( a,b ) $: 因为密度函数为
        \begin{equation*}
            p (x) = 
            \begin{cases}
                \dfrac{1}{b-a}, & a < x < b,\\
                0, & \text{其他}.
            \end{cases}
        \end{equation*}
        所以特征函数为
        \begin{equation*}
            \varphi (t) = \int_a^b \frac{\ee^{itx}}{b - a} \dd x = \frac{\ee^{ibt} - \ee^{iat}}{it (b-a)}.
        \end{equation*}
        \item
        \textbf{标准正态分布}\index{标准正态分布} $ N (0,1) $: 因为密度函数为
        \begin{equation*}
            p (x) = \frac{1}{\sqrt{2\pi}} \exp \left( -\frac{x^2}{2} \right), \quad - \infty < x < + \infty.
        \end{equation*}
        所以特征函数为
        \begin{align*}
            \varphi (t) & = \frac{1}{\sqrt{2\pi}} \int_{-\infty}^{+\infty} \exp \left( itx - \frac{x^2}{2} \right) \dd x\\
            & = \exp \left( -\frac{t^2}{2} \right) \frac{1}{\sqrt{2\pi}} \int_{-\infty}^{+\infty} \exp \left( -\frac{( x - it )^2}{2} \right) \dd x\\
            & = \exp \left( -\frac{t^2}{2} \right) \frac{1}{\sqrt{2\pi}} \int_{-\infty -it}^{+\infty -it} \exp \left( -\frac{x^2}{2} \right) \dd x\\
            & = \exp \left( -\frac{t^2}{2} \right),
        \end{align*}
        其中
        \begin{equation*}
            \int_{-\infty -it}^{+\infty -it} \exp \left( -\frac{x^2}{2} \right) \dd x = \sqrt{2\pi}
        \end{equation*}
        是利用复变函数中的围道积分求得的.
        有了标准正态分布的特征函数,
        再利用下节给出的特征函数的性质,
        就很容易得到一般正态分布 $ N ( \mu, \sigma^2 ) $ 的特征函数,
        见例~\ref{exam:4.1.2}.
        \item
        \textbf{指数分布}\index{指数分布} $ Exp ( \lambda ) $: 因为密度函数为
        \begin{equation*}
            p (x) =
            \begin{cases}
                \lambda \ee^{-\lambda x}, & x > 0,\\
                0, & x \leq 0.
            \end{cases}
        \end{equation*}
        所以特征函数为
        \begin{align*}
            \varphi (t) & = \int_0^{+\infty} \ee^{itx} \lambda \ee^{-\lambda x} \dd x\\
            & = \lambda \left( \int_0^{+\infty} \cos (tx) \ee^{-\lambda x} \dd x + i \int_0^{+\infty} \sin (tx) \ee^{-\lambda x} \dd x \right)\\
            & = \lambda \left( \frac{\lambda}{\lambda^2 + t^2} + i \frac{t}{\lambda^2 + t^2} \right)\\
            & = \left( 1 - \frac{it}{\lambda} \right)^{-1}.
        \end{align*}
        以上积分中用到了复变函数中的欧拉公式: $ \ee^{itx} = \cos (tx) + i \sin ( tx) $.
    \end{enumerate}
\end{example}

\subsection{特征函数的性质}

现在我们来研究特征函数的一些性质,
其中 $ \varphi_X (t) $ 表示 $ X $ 的特征函数,
其他类似.

\begin{property}\label{prop:4.1.1}
    \begin{equation}\label{eq:4.1.4}
        \lvert \varphi (t) \rvert \leq \varphi (0) = 1.
    \end{equation}
\end{property}

\begin{property}\label{prop:4.1.2}
    \begin{equation}\label{eq:4.1.5}
        \varphi (-t) = \overline{\varphi (t)},
    \end{equation}
    其中 $ \overline{\varphi (t)} $ 表示 $ \overline{\varphi (t)} $ 的共轭.
\end{property}

\begin{property}\label{prop:4.1.3}
    若 $ Y = aX + b $, 其中 $ a,b $ 是常数, 则
    \begin{equation}\label{eq:4.1.6}
        \varphi_Y (t) = \ee^{ibt} \varphi_X (at).
    \end{equation}
\end{property}

\begin{property}\label{prop:4.1.4}
    独立随机变量和的特征函数为特征函数的积,
    即设 $ X $ 与 $ Y $ 相互独立, 则
    \begin{equation}\label{eq:4.1.7}
        \varphi_{X+Y} (t) = \varphi_X (t) \cdot \varphi_Y (t).
    \end{equation}
\end{property}

\begin{property}\label{prop:4.1.5}
    若 $ E (x^l) $ 存在,
    则 $ X $ 的特征函数 $ \varphi(t) $ 可 $ l $ 次求导,
    且对 $ 1 \leq k \leq l $, 有
    \begin{equation}\label{eq:4.1.8}
        \varphi^{(k)} (0) = i^k E ( X^k ).
    \end{equation}
    上式提供了一条求随机变量的各阶矩的途径,
    特别可用下式去求数学期望和方差.
    \begin{equation}\label{eq:4.1.9}
        E (X) = \frac{\varphi' (0)}{i}, \quad \mathrm{Var} (X) = - \varphi'' (0) + \bigl( \varphi' (0) \bigr)^2.
    \end{equation}
\end{property}

\begin{proof}
    在此我们仅对连续场合进行证明,
    而在离散场合的证明是类似的.
    \begin{enumerate}
        \item 
        \begin{align*}
            \lvert \varphi (t) \rvert & = \left\lvert \int_{-\infty}^{+\infty} \ee^{itx} p (x) \dd x \right\rvert 
            \leq \int_{-\infty}^{+\infty} \left\lvert \ee^{itx} \right\rvert p (x) \dd x\\
            & = \int_{-\infty}^{+\infty} p (x) \dd x
            = \varphi (0)
            = 1.
        \end{align*}
        \item 
        \begin{equation*}
            \varphi (-t) = \int_{-\infty}^{+\infty} \ee^{-itx} p (x) \dd x
            = \overline{\int_{-\infty}^{+\infty} \ee^{itx} p (x) \dd x}
            = \overline{\varphi (t)}.
        \end{equation*}
        \item
        \begin{equation*}
            \varphi_Y (t) = E ( \ee^{it (aX + b)} )
            = \ee^{ibt} E ( \ee^{iatX} ) = \ee^{ibt} \varphi (at).
        \end{equation*}
        \item
        因为 $ X $ 与 $ Y $ 相互独立, 所以 $ \ee^{itX} $ 与 $ \ee^{itY} $ 也是独立的, 从而有
        \begin{equation*}
            E \left( \ee^{it ( X + Y )} \right) = E \left( \ee^{itX} \ee^{itY} \right) = \varphi_X (t) \cdot \varphi_Y (t).
        \end{equation*}
        \item
        因为 $ E \left( X^l \right) $ 存在, 也就是
        \begin{equation*}
            \int_{-\infty}^{+\infty} \lvert x \rvert^l p (x) \dd x < +\infty,
        \end{equation*}
        于是含参变量 $ t $ 的广义积分 $ \int_{-\infty}^{+\infty} \ee^{itx} p(x) \dd x $ 可以对 $ t $ 求导 $ l $ 次, 于是对 $ 0 \leq k \leq l $, 有
        \begin{equation*}
            \varphi ^{(k)} (t) = \int_{-\infty}^{+\infty} i^k x^k \ee^{itx} p (x) \dd x = i^k E \left( X^k \ee^{itX} \right).
        \end{equation*}
        令 $ t = 0 $ 即得
        \begin{equation*}
            \varphi^{(k)} (0) = i^k E \left( X^k \right).
        \end{equation*}
    \end{enumerate}
    至此上述5条性质全部得证.
\end{proof}

下例是利用 \eqref{eq:4.1.6} 和 \eqref{eq:4.1.7} 来求一些常用分布的特征函数.

\begin{example}{常用分布的特征函数 (二)}\label{exam:4.1.2}
    \begin{enumerate}
        \item
        \textbf{二项分布}\index{二项分布} $ b (n, p) $: 设 $ Y \sim b (n, p) $, 则 $ Y = X_1 + x_2 + \dotsb X_n $, 其中诸 $ X_i $ 是相互独立同分布的随机变量, 且 $ X_{i} \sim b (1, p) $, 由 例~\ref{exam:4.1.2} \ref{exam:4.1.1.2} 知
        \begin{equation*}
            \varphi_{X_i} (t) = p \ee^{it} + q,
        \end{equation*}
        所以由独立随机变量和的特征函数为特征函数的积, 得
        \begin{equation*}
            \varphi_Y (t) = \left( p \ee^{it} + q \right)^{\pi}
        \end{equation*}
        \item
        \textbf{正态分布}\index{正态分布} $ N (\mu, \sigma^2) $: 设 $ Y \sim N (\mu, \sigma^2) $, 则 $ X = (Y - \mu) / \sigma \sim N (0, 1) $.
        由例~\ref{exam:4.1.1} 知
        \begin{equation*}
            \varphi_{X} (t) = \exp \left( -\frac{t^2}{2} \right).
        \end{equation*}
        所以由 $ Y = \sigma X + \mu $ 得
        \begin{equation*}
            \varphi_Y (t) = \varphi_{\alpha X + \mu} (t) =\ee^{i \mu t} \varphi_X ( \sigma t ) = \exp \left( i \mu t - \frac{\sigma^2 t^2}{2} \right).
        \end{equation*}
        \item 
        \textbf{伽玛分布}\index{伽玛分布} $ Ga ( n, \lambda ) $: 设 $ Y \sim Ga ( n, \lambda ) $, 则$ Y = X_1 + X_2 + \dotsb + X_n $, 其中 $ X_i $ 独立同分布, 且 $ X_i \sim Exp ( \lambda ) $.
        由例~\ref{exam:4.1.1} 知
        \begin{equation*}
            \varphi_{X_i} (t) = \left( 1 - \frac{it}{\lambda} \right)^{-1}.
        \end{equation*}
        所以由独立随机变量和的特征函数为特征函数的积, 得
        \begin{equation*}
            \varphi_Y (t) = \left( \varphi_{X_i} (t) \right)^n = \left( 1 - \frac{it}{\lambda} \right)^{-n}.
        \end{equation*}
        进一步, 当 $ a $为任一正实数时, 我们可得 $ Ga ( n, \lambda ) $ 分布的特征函数为
        \begin{equation*}
            \varphi(t) = \left( 1 - \frac{it}{\lambda} \right)^{-a}.
        \end{equation*}
        \item
        $ \mathbf{\chi^2 (n)} $ \textbf{分布}\index{$ \mathbf{\chi^2 (n)} $ 分布}: 因为 $ \chi^2 (n) = Ga ( n/2, 1/2 ) $, 所以 $ \chi^2 (n) $ 分布的特征函数为
        \begin{equation*}
            \varphi (t) = ( 1 - 2it )^{-n/2}.
        \end{equation*}
    \end{enumerate}
\end{example}

上述常用分布的特征函数汇总在表~\ref{tab:4.1.1} 中.

\begin{table}
    \renewcommand{\arraystretch}{1.6}
    \centering
    \caption{常用分布的特征函数}\label{tab:4.1.1}
    \begin{tabular}{>{\centering\arraybackslash}m{0.15\linewidth}>{\centering\arraybackslash}m{0.45\linewidth}>{\centering\arraybackslash}m{0.3\linewidth}}
        \toprule
        分布 & 分布列 $ p_k $ 或分布密度 $ p (x) $ & 特征函数 $ \varphi (t) $\\
        \midrule
        单点分布 & $ P (X = a) = 1 $ & $ \ee^{itz} $\\
        0-1分布 & $ P_k = p^k  q^{1 - k} $, $ k = 0,1 $ & $ p \ee^{it} + q $\\
        二项分布 $ b (n, p) $ & $ p_k = \binom{n}{k} p^k q^{1-k} $, $ k = 0,1,\dotsc,n $ & $ \left( p \ee^{it} + q \right)^{\pi} $\\
        泊松分布 $ P (\lambda) $ & $ P_k = ( \lambda^k/k! ) \ee^{-\lambda} $, $ k = 0, 1, \dotsc $ & $ \ee^{\lambda ( \ee^{it} ) -1} $\\
        均匀分布 $ U (a,b) $ & $ p (x) = 1/(b-a) $, $ a \leq x \leq b $ & $ \dfrac{\ee^{ibt} - \ee^{iat}}{it (b-a)} $\\
        正态分布 $ N (\mu, \sigma^2) $ & $ p (x) = \dfrac{1}{\sqrt{2\pi}\sigma} \exp \left( - \dfrac{( x - \mu )^2}{2 \sigma^2} \right) $ & $ \exp \left( i \mu t - \dfrac{\sigma^2 t^2}{2} \right) $\\
        指数分布 $ Exp ( \lambda ) $ & $ p (x) = \lambda \ee^{-\lambda x} $, $ x > 0 $ & $ \left( 1 - \dfrac{it}{\lambda} \right)^{-1} $\\
        伽玛分布 $ Ga ( a, \lambda ) $ & $ p (x) = \dfrac{\lambda^\alpha}{\Gamma (a)} x^{\alpha - 1} \ee^{-\lambda x} $, $ x \geq 0 $ & $ \left( 1 - \dfrac{it}{\lambda} \right)^{-a} $\\
        $ \chi^2 (n) $ 分布 & $ p (x) = \dfrac{x^{n/2 - 1} \ee^{-x/2}}{\Gamma (n/2) 2^{n/2}} $, $ x > 0 $ & $ ( 1 - 2it )^{-n/2} $\\
        \bottomrule
    \end{tabular}
\end{table}

下例是利用 \eqref{eq:4.1.8} 来求分布的数学期望和方差.

\begin{example}\label{exam:4.1.3}
    试利用特征函数的方法求伽玛分布 $ Ga ( n, \lambda ) $ 的数学期望和方差.
\end{example}

\begin{solution}
    因为伽玛分布 $ Ga ( a, \lambda ) $ 的特征函数及其一、二阶导数为
    \begin{gather*}
        \varphi (t) = \left( 1 - \dfrac{it}{\lambda} \right)^{-a},\\
        \varphi' (t) = \frac{ai}{\lambda} \left( 1 - \frac{it}{\lambda} \right)^{-a-1}, \ \varphi' (0) = \frac{ai}{\lambda},\\
        \varphi'' (t) = \frac{a ( a + 1 ) i^2}{\lambda^2} \left( 1 - \frac{it}{\lambda} \right)^{-a-2}, \ \varphi'' (0) = -\frac{a (a+1)}{\lambda^2},
    \end{gather*}
    所以由 \eqref{eq:4.1.9} 得
    \begin{align*}
        E (X) & = \frac{\varphi' (0)}{i} = \frac{a}{\lambda},\\
        Var (X) & = - \varphi'' (0) + \bigl( \varphi' (0) \bigr)^2 = \frac{a (a+1)}{\lambda^2} + \left( \frac{ai}{\lambda} \right)^2\\
        & = \frac{a (a+1)}{\lambda^2} - \frac{a^2}{\lambda^2} = \frac{a}{\lambda^2}.
    \end{align*}
\end{solution}
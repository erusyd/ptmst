% Edit by:xiangyu
\chapter{统计量及其分布}
前四章的研究属于概率论的范畴。 我们已经看到,随机变量及其概率分布全面地描述了随机现象的通解规律性。在概率论的许多问题中,概率分布通常被假定为已知的,而一切计算及推理均基于这个已知的分布进行,在实际问题中,情况往往并非如此,看一个例子。
\begin{example}\label{li5.0.1}
某公司要采购一批产品,每件产品不是合格品就是不合格品,但该产品总有一个不合格品率$p$。由此,若从该批产品中随机抽取一件,用$X$表示这一件产品的不合格数,不难看出$X$服从一个二点分布$b(1,p)$,但分布中的参数$p$却是不知道的。显然, $p$的大小决定了该批产品的质量,它直接影响采购行为的经济效益。因此,人们会对$p$提出一些问题,比如,
\begin{itemize}
\item $p$的大小如何;
\item $p$大概落在什么范围内;
\item 能否认为$p$满足设定要求(如$p\le0.05$)。
\end{itemize}
\end{example}
诸如例\ref{li5.0.1}研究的问题属于数理统计的范畴。接下来我们从统计中最基本的概念——总体和样本开始介绍统计学内容。
\section{总体与样本}
\subsection{总体与个体}
在一个统计问题中,我们把研究对象的全体称为\textbf{总体}\index{Z!总体},构成总体的每个成员称为\textbf{个体}\index{G!个体}。对多数实际问题,总体中的个体是一些实在的人或物。比如,我们要研究某大学的学生身高情况,则该大学的全体学生构成问题的总体
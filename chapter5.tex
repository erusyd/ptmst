% Edit by:xiangyu
\chapter{统计量及其分布\label{cha:5}}
前四章的研究属于概率论的范畴. 我们已经看到, 随机变量及其概率分布全面地描述了随机现象的通解规律性.在概率论的许多问题中, 概率分布通常被假定为已知的, 而一切计算及推理均基于这个已知的分布进行, 在实际问题中, 情况往往并非如此, 看一个例子.
\begin{example}\label{exam5.0.1}
某公司要采购一批产品, 每件产品不是合格品就是不合格品, 但该产品总有一个不合格品率$p$.由此, 若从该批产品中随机抽取一件, 用$X$表示这一件产品的不合格数, 不难看出$X$服从一个二点分布$b(1,p)$,但分布中的参数$p$却是不知道的.显然,  $p$的大小决定了该批产品的质量, 它直接影响采购行为的经济效益.因此, 人们会对$p$提出一些问题, 比如,
\begin{itemize}
\item $p$的大小如何;
\item $p$大概落在什么范围内;
\item 能否认为$p$满足设定要求(如$p\leq0.05$).
\end{itemize}
\end{example}
诸如例~\ref{exam5.0.1} 研究的问题属于数理统计的范畴.接下来我们从统计中最基本的概念——总体和样本开始介绍统计学内容.
\section{总体与样本\label{sec:8.1}}
\subsection{总体与个体\label{5.1.1}}
在一个统计问题中, 我们把研究对象的全体称为\textbf{总体}\index{Z!总体}, 构成总体的每个成员称为\textbf{个体}\index{G!个体}.对多数实际问题, 总体中的个体是一些实在的人或物.比如, 我们要研究某大学的学生身高情况, 则该大学的全体学生构成问题的总体, 而每一个学生即使一个个体.事实上, 每个学生都有许多特征:性别、年龄、身高、体重、名字、籍贯等等, 而在该问题中, 我们关心的只是该校学生的身高如何, 对其他的特征暂不予考虑.这样, 每个学生(个体)所具有的数量指标值——身高就是个体, 而将所有身高全体看成总体.这样一来, 若抛开实际背景, 总体就是一堆数, 这堆数中有大有小, 有的出现的机会多, 有的出现机会小, 因此用一个概率分布去描述和归纳总体是恰当的, 从这个意义看, \textbf{总体就是一个分布}, 而其数量指标就是服从这个分布的随机变量.以后说``从总体中抽样"与``从分布中抽样"是同一个意思.
\begin{example}
考察某厂的产品质量, 将其产品只分为合格品与不合格品, 并以$0$记合格品, 以$1$记不合格品, 则
\[\text{总体}=\{\text{该厂生产的全部合格品与不合格品}\}=\{\text{由}~0~\text{或}~1~\text{组成的一堆数}\}.\]
若以$p$表示这堆数中$1$的比例(不合格品率), 则该总体可由一个二点分布表示:
\begin{center}
\begin{tabularx}{0.4\textwidth}{Y|YY}
  X&0&1\\
  \midrule
  P&1-p&p
  \end{tabularx}
\end{center}
不同的$p$反映了总体间的差异. 譬如,两个生产同类产品的工厂的产品总体分布为

\begin{minipage}{0.4\textwidth}
\centering
\begin{tabularx}{\textwidth}{Y|YY}
X&0&1\\
\midrule
P&0.983&0.017
\end{tabularx}
\end{minipage}\hspace{3\ccwd}
\begin{minipage}{0.4\textwidth}
\centering
\begin{tabularx}{\textwidth}{Y|YY}
X&0&1\\
\midrule
P&0.915&0.085
\end{tabularx}
\end{minipage}
我们可以看到,第一个工厂的产品质量优于第二个工厂.实际中,分布中的不合格率是未知的,如何对之进行估计是统计学要研究的问题.
\end{example}
\begin{example}
彩电的彩色浓度是彩电质量好坏的一个重要指标. 20世纪70年代在美国销售的SONY牌彩电有两个产地:美国和日本,两地的工厂是按统一设计方案和相同的生产线生产同一型号SONY彩电,连使用说明书和检验合格的标准也是一样的.其中关于彩色浓度X的标准是:目标值为$m$,公差为$5$,即当$X$在$[m-5,m+5]$内该彩电的热情高于购买美产SONY彩电,原因何在?这就要考察这两个总体有什么差别. 1979年4月17日日本《朝日新闻》刊登调查报告指出,日产SONY彩电的彩色浓度服从正态分布$N(m,(5/3)^2)$,而美产SONY彩电的彩色浓度服从$(m-5,m+5)$上的均匀分布,见图~\ref{table5.1.1}.这两个不同的分布代表了不同的总体,其均值相同(都为$m$),但方差不同.若彩色浓度与$m$的距离在$5/3$以内为\Rmnum{1}级品,在$5/3$到$10/3$之间为\Rmnum{2}级品,在$10/3$到$5$之间为\Rmnum{3}级品,其他为\Rmnum{4}级品.于是日产SONY彩电的\Rmnum{1}级品为美产SONY的两倍出头(见表~\ref{table5.1.1}),这就是美国消费者愿意购买日产SONY的主要原因.
\end{example}

在有些问题中,我们对每一研究对象可能要观测两个甚至更多个指标,此时可用多维随机向量及其联合分布来描述总体.这种总体称为多维总体.譬如,我们要了解某校大学生的三个指标: 年龄、身高、月生活支出.则我们可用一个三维随机向量描述该总体.这是一个三维总体,它是多元分析所研究的对象.本书中主要研究一维总体,某些地方也会涉及二维总体.

总体还有有限总体和无限总体,本书将以无限总体作为主要研究对象.
\subsection{样本}
为了了解总体的分布,我们从总体中随机地抽取$n$个个体,记其指标值为$x_1,x_2,\cdots,x_n$,则$x_1,x_2,\cdots,x_n$称为总体的一个\textbf{样本}\index{Y!样本}, $n$称为\textbf{样本容量}\index{Y!样本容量},或简称为\textbf{样本量}\index{Y!样本量},样本中的个体称为\textbf{样品}\index{Y!样品}.